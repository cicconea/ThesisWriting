%  1. To get appendices, you don't do anything different from a normal
%     report document.  That means, put the command \appendix before
%     you begin your first appendix, then do each appendix with a
%     \chapter command.  Note that if you have only one appendix, it is
%     customary to leave it unnumbered.  Do this with \chapter*.

%  5. If your References section doesn't show up in the table of
%     contents, you need to add the line \addcontentsline...
%     as done at end of this file.   Make sure that you include the 
%     page break command as done there or else you may end up
%     with the wrong page number in your table of contents. 
 
\documentclass[singlespace]{easychithesis}
\usepackage[english]{babel}
\usepackage[utf8]{inputenc}
\usepackage{amsmath}
\usepackage{graphicx}
\usepackage{amssymb}
\newcommand{\Lagr}{\mathcal{L}}
\usepackage{tabulary}
\usepackage{float}
\floatstyle{boxed}
\restylefloat{figure}
\usepackage[section]{placeins}




\begin{document}

\title{A Simple Model of Energy Transition Under a Carbon Budget} 
\author{Adriana Ciccone}
\date{\today}
\department{Environmental Science \& Policy}
\division{The Harris School of Public Policy Studies} 
\degree{Master of Science} 
\maketitle

% \dedication : Use for a dedication, copyright, or epigraph.
%               Produces a page with no number for the text which follows
%               If you want centering, do it yourself with 
%               \begin{center} and \end{center}.  You can have more
%               than one `dedication'.
%\dedication
%\begin{center}
%        To Roger Dean
%\end{center}


% \topmatter : Things like Abstract, Acknowledgements.
% For the abstract, you can also do 
%      \begin{abstract} ...text... \end{abstract}
% if you prefer.

\topmatter{Abstract}
TBD


%\topmatter{Acknowledgements}

%
% Table Of Contents
%

\tableofcontents

%
% Begin Body
%
\mainmatter

\chapter{Introduction}
\section{Relevance}

View problem of climate change as one of transition from fossil fuel energy to clean energy.

Need to replace large durable stock of infrastructure while meeting energy demand.

Goal is to understand the speed of transition.

\begin{itemize}
	\item{Do we keep on installing fossil fuel plants and plan on a quick transition in the future or do we gradually phase out fossil fuels?}
	\item{What does the timing depend on?}
	\item{If the optimal timing is to start replacing now, how does that affect treaty negotiations?}
\end{itemize}



Question: how to best transition to clean energy

\begin{itemize}
\item Reducing emissions involves replacing the large, durable fossil fuel
infrastructure.
\item Fossil fuel energy is cheaper than renewables.
\item But the cost of renewables is going down over time.
\item What is the best transition path?


\begin{itemize}
\item Gradually replace fossil energy as plants and equipment depreciate?
\item Keep replacing fossil with more fossil and then switch quickly?
\item Gradual transition means no early retirements but at the cost of
installing clean energy earlier (so maybe more expensive).
\item Waiting means future early retirements but cheaper clean energy.
\end{itemize}
\end{itemize}




\begin{itemize}
\item Simplest possible model that captures the dynamics.

\item Develop intuitions, orders of magnitude effects.

\item Allows us to consider uncertainty, robustness etc.

\item Not attempting to model actual energy system.
\end{itemize}



Exogenous:

	\begin{itemize}
    
    	\item{Emissions Cap}
        \item{Energy demand}

	\end{itemize}


Have to replace fossil fuel with clean energy to stay within the cap while meeting the energy demand.

Cost of clean energy goes down over time. 

Timing choice: 

  \begin{itemize}

	\item{Install more expensive clean energy now}
	\item{Or wait for cheaper clean energy at the cost of installing new fossil fuel plants and retiring them early.}

\end{itemize}



\section{Literature Review}

\begin{itemize}
\item Engineering approaches

\begin{itemize}
\item Try to determine precise number of windmills, solar panels and so
forth.

\item E.g., Work by Jacobson.

\item E.g. Earth Institute Deep Decarbonization Report
\end{itemize}

\item Jacard and Rivers

\begin{itemize}
\item Optimal stopping problem.
\end{itemize}
\end{itemize}





%%%%%%%%%%%%%%%%%%%%%%%%%%%%%%%%%%%%%%%
%%%%%%%%%%%%%%%%%%%%%%%%%%%%%%%%%%%%%%%

\chapter{Modeling Choices}
\section{Model Setup}

Power generation is accomplished by dollars of capital ($k^h(t), k^l(t)$) that can produce certain amounts of electricity in a given year.Capital is comprised of an initial investment $H_0$ or $L_0$ that depreciates exponentially, but can be supplemented with positive investments $H^+(t)$ or $L^+(t)$ that also depreciate exponentially, or retired early with $H^-(t)$ or $L^-(t)$.  \\

Power is generated by multiplying each type of capital in a year by the overnight capital cost scaled by a capacity factor ($F^h(t), F^l(t)$. High-emitting (dirty) capital has a lower capital cost than low-emitting (clean) capital, but also has higher emissions ($m^h(t) > m^l(t) \forall t$). \\

To minimize the cost of early retirement, we set our objective function to minimize the net present value of new investments ($H^+(t)$ and $L^+(t)$). Any early retirement of existing capital stock removes it from the generating capacity, so in order to meet power demand requirements, new capital must be built, which is penalized by the objective function. 



%%%%%%%%%%%%%%%%%%%%%%%%%%%%%%%%%%%%%%%%%%%%%%
\section{Model Description}\label{sec:ModelDesc}

\paragraph{} The region or country of interest produces electricity by two methods: clean (low emitting, represented by $k^l(t)$) and dirty (high emitting, represented by $k^h(t)$). Clean capital produces electricity with low carbon dioxide emissions, but, as a newer technology, it is more expensive per unit of power output than dirty. Since it is a new technology, the capital cost is expected to decrease over time as the production process matures and technological innovation produces lower-cost units. Dirty capital is a mature technology that is cheap to build but emits a higher level of carbon dioxide per unit of electricity produced than clean. Since it is a mature technology, there is little reason to believe that production costs will decline substantially in the future. 


\paragraph{} Each type of capital depreciates exponentially over time. The decision variables for the optimization are additional investments in each type of capital that also depreciate at the same rate, represented by four functions ($H^+(t), H^-(t), L^+(t), \& L^-(t)$). These decision variables define the optimal path of investment in dirty (high emitting) and clean (low emitting) capital. All four functions are non-negative over their domain. The functions $H^+(t)$ and $L^+(t)$ represent additional dollars of investment to build new capital of each type. $H^-(t)$ and $L^-(t)$ remove capital from the economy via early retirement. There is no scrap cost to early retirement - the economy simply loses the ability to produce electricity from that particular facility. As I show in the next section, this has consequences for the ability of the economy to meet it's total electricity demand. The retired plant cannot be re-started. Any need for additional capital in time t must be met by a positive investment in either $H^+(t)$ or $L^+(t)$. 

\begin{equation}\label{eq:simpleHCapitalConstraint}
k^h(t) = H_0 e^{-t/n} + \int_0^t e^{x-t/n} H^+(x) dx - \int_0^t H^-(x)dx \geq 0 \qquad \forall t
\end{equation}

\begin{equation}\label{eq:simpleLCapitalConstraint}
k^l(t) = L_0 e^{-t/n} + \int_0^t e^{x-t/n} L^+(x) dx - \int_0^t L^-(x)dx \geq 0 \qquad \forall t
\end{equation}

These equations express total capital in each time period as a function of some exponential depreciation and all prior investments. At each time t, the initial capital endowment, $H_0 = H^+(0)$ and $L_0 = L^+(0)$ depreciate exponentially and any further positive investments depreciate based on their age (x-t). The integral in the above expression shows how each investment $H^+(t)$ or $L^+(t)$ is weighted by an exponential term of the difference in that investment's age and the current time period. Therefore we see that early investments have a more negative exponential term and count less towards existing capital than more recent investments. While we have no limits on the amount of capital in each time period other than the non-negativity constraint on the decision variables, a physical limitation remains that we cannot have negative amounts of capital. We therefore restrict both $k^h(t)$ and $k^l(t)$ to be non-negative in all time periods. 


\paragraph{} The objective is to minimize total additional positive investment in both high and low carbon intensity capital for the lowest cost transition to clean energy. We minimize over positive investment only to avoid negative investment - early retirement - benefiting the objective. Instead, early retirement forces additional investment later in the simulation due to an energy generation constraint. That is, if we retire a certain amount of capital early and have a shortfall in generation capacity, we must make a positive investment at some later point. This extra generation would be counted as additional positive investment in the below objective function, penalizing early retirement. The sum is discounted by an interest rate so that we minimize over the net present value of investments. 

\begin{equation}\label{eq:simpleObjective}
C(H^+(t), L^+(t)) = \int_0^N (H^+(t) + L^+(t))e^{-rt}) dt \quad if H(t), L(t) \geq 0
\end{equation}



%% No operating cost, so when we early retire, we leave the pile of coal that we had purchased upfront - biases the result up/down? Clean doesn't have this problem %%


\paragraph{} The economy has some total energy demand that can follow any exogenous path over time. This represents total annual electricity demand, in billions of kilowatt-hours, and therefore daily and seasonal fluctuations in power demand are ignored. Capital must be used to produce electricity to meet this demand, and we must convert from dollars of capital to kilowatt-hours of electricity to take in to account the relative cost difference of producing one unit of clean energy versus one unit of dirty energy. 

\begin{equation}\label{eq:simpleGenConstraint}
G(t) = F^h(t) k^h(t) + F^l(t) k^l(t)
\end{equation}

In this equation, $F^h (t)$ and $F^l (t)$ are the overnight capital cost per billion kilowatt-hours per year. That is, they represent the number of kilowatt-hours per year that each dollar of capital can produce. In early years when the clean technology is still relatively new and expensive, a dollar of capital may not produce much energy per year, but a dollar of clean capital invested later in the model may produce significantly more energy. It is important to be clear however, on what this cost is. It is not an annual maintenance cost or any measure of financial obligation for additional production. It is simply the the ability of that unit (dollar) of capital to produce a certain amount of electricity. 

%% We don't track vintages over time, so these costs are just an assumed exogenous trajectory. Which bias? Also we take care of this in the vintaged version%%


\paragraph{} An emissions cap is negotiated by the region's government. Over the course of a fixed number of years, the region must not emit more than a negotiated percent of business as usual emissions. In practice, the government could negotiate to agree to a limit on total number of tons of $CO_2$ emitted instead of negotiating a percentage of business as usual. However, it is convenient to represent a bulk emissions cap as a percent reduction for both modeling and to understand how our optimal cost will change as a function of total allowed emissions.

\begin{equation}\label{eq:simpleMaxEmission}
E_{max} = \alpha \int_0^N m^h(t) F^h(t) H^+(0) + m^l(t) F^l(t) L^+(0) dt \qquad \alpha \in [0,1]
\end{equation}

To arrive at an emissions cap, we transform dollars of capital to kilowatt-hours per year as above, then multiply by the carbon intensity of energy, $m^h(t)$ and $m^l(t)$. This value is a measurement of the pounds of $CO_2$ emitted from power generation per kilowatt-hour. We can then write actual emissions over the length of simulation as the following. 

\begin{equation}\label{eq:simpleSimEmission}
E(H, L) = \int_0^N  m^h(t) F^h(t) H_0 + m^l(t) F^l(t) L_0 dt
\end{equation}

Total emissions over the treaty's length must not exceed the cap represented by a fraction of business as usual, leading to the final constraint: 

\begin{equation}\label{eq:simpleEmitConstraint}
E(H,L) \leq E_{max}
\end{equation}



%%%%%%%%%%%%%%%%%%%%%%%%%%%%%%%%%%%%%%%%%%%%%%%%%%%%%%%%%%%%%
\subsection{Solving the Model Analytically}

In order to attempt an analytical solution, I make several key assumptions about the functional forms of key parameters. While the equations for capital stocks remain the same, I assume a flat, fixed energy demand, $G$, through the length of the simulation. The energy per unit of capital for high-emitting technologies is similarly fixed at $F^h$ for the entire period, while it decays exponentially for low-emitting technology with a half life of $\tau$. Assuming these functional forms is arguably a good approximation of existing beliefs about the future costs of low-emitting technologies with learning by doing or economies of scale; that is, a rapid improvement in the output per dollar of capital that slowly levels off as t grows large. I further assume that high-emitting carbon intensity is fixed at $m^h$ and that there are no emissions associated with clean capital. With these simplification we now write the following optimization problem. \\

Objective: Chose $H^+(t)$, $H^-(t)$, $L^+(t)$, and $L^-(t)$ to minimize:

\begin{equation}\label{eq:analyticalObj}
C = \int_0^N (H^+(t) + L^+(t))e^{-rt} dt
\end{equation}

Subject to: 

\begin{multline}\label{eq:analyticalGen}
G = \bar{F}^h\left (H_0 e^{-t/n} + \int_0^t e^{x-t/n} H^+(x) dx - \int_0^t H^-(x)dx \right) \\ + \bar{F}^l(1-e^{-t/\tau}) \left ( L_0 e^{-t/n} + \int_0^t e^{x-t/n} L^+(x) dx - \int_0^t L^-(x)dx\right )
\end{multline}

\begin{equation}\label{eq:analyticalEmit}
m^hF^h\left (H_0 e^{-t/n} + \int_0^t e^{x-t/n} H^+(x) dx - \int_0^t H^-(x)dx \right) \leq \alpha N m^h F^h H^+(0)
\end{equation}

\begin{equation}\label{eq:analyticalHCapConstraint}
H_0 e^{-t/n} + \int_0^t e^{x-t/n} H^+(x) dx - \int_0^t H^-(x)dx \geq 0 
\end{equation}

\begin{equation}\label{eq:analyticalLCapConstraint}
L_0 e^{-t/n} + \int_0^t e^{x-t/n} L^+(x) dx - \int_0^t L^-(x)dx \geq 0 
\end{equation}

\begin{equation}\label{eq:analyticalDecisionVarConstraint}
H^+(t), H^-(t), L^+(t), L^-(t) \geq 0
\end{equation}

%%%%%%%%%%%%%%%%%%%%%%%%%%%%%%%%%%%%%%%%%%%%%%%%%%%%%%%%%%%%

\subsection{First Order Conditions for Simple Model}

Note that a solution will go here if I can get one. For now, the first order conditions are as follows:

\begin{equation}\label{eq:dLdH+}
\frac{d\Lagr}{dH^+(t)} = 1/r(e^{-rN} -1) + n(1-e^{-t/n})(\rho - \lambda F^h -\mu m F^h) + \sigma = 0
\end{equation}

\begin{equation}\label{eq:dLdH-}
\frac{d\Lagr}{dH^-(t)} = (\lambda F^h +\mu m F^h - \rho)t + \phi = 0
\end{equation}

\begin{equation}\label{eq:dLdL+}
\frac{d\Lagr}{dH^+(t)} = 1/r(e^{-rN} -1) + n(1-e^{-t/n})(\pi - \lambda F^l(1-e^{-t/\tau})) + \gamma = 0
\end{equation}

\begin{equation}\label{eq:dLdL-}
\frac{d\Lagr}{dH^-(t)} = (\lambda F^l(1-e^{-t/\tau}) - \pi)t + \delta = 0
\end{equation}

\begin{multline}\label{eq:dLdlambda}
\frac{d\Lagr}{d\lambda} = G - F^h\left[nH^+(t) + e^{-t/n}(n^2 + H_0(1-n)) - n^2 + \int_0^tH^-(x)dx\right] \\ - F^l (1-e^{-t/\tau})\left[nL^+(t) + e^{-t/n}(n^2 + L_0(1-n)) - n^2 + \int_0^tL^-(x)dx\right] = 0
\end{multline}

\begin{multline}\label{eq:dLdmu}
\frac{d\Lagr}{d\mu} = \bar{E} - mF^h \left [(1-e^{-N/n})(1-n)H_0 n + n^2(N + n(1-e^{-N/n})) \right.\\ + \left. n\int_0^N H^+(t) dt - \int_0^N\int_0^t H^-(x)dx dt   \right] \geq 0
\end{multline}

\begin{equation}\label{eq:dLdrho}
\frac{d\Lagr}{d\rho} = H_0 e^{-t/n}(1-n) + nH^+(t) - n^2(1-e^{-t/n}) - \int_0^t H^-(x)dx \geq 0 
\end{equation}

\begin{equation}\label{eq:dLdpi}
\frac{d\Lagr}{d\pi} = L_0 e^{-t/n}(1-n) + nL^+(t) - n^2(1-e^{-t/n}) - \int_0^t L^-(x)dx \geq 0 
\end{equation}





%%%%%%%%%%%%%%%%%%%%%%%%%%%%%%%%%%%%%%
\chapter{Results}
\section{Numerical Simulation}

\subsection{Implementation}
The code for the following results was written in Python using the Pyomo modeling and optimization library. All data and code can be found at \\ https://github.com/cicconea/EnergyTransition. Data for parameters comes from the Energy Information Administration (EIA) unless otherwise stated.  

\subsection{Functional Form Choices}

\paragraph{} Several key parameters are allowed to vary over time. In this section I explain the modeling choices and sources of data for each of these parameters. 

\paragraph{} Energy demand is exogenous and comes directly from the EIA. From the EIA demand forecast, I subtract generating capacity supplied by hydroelectric dams and nuclear plants to evaluate only the subset of demand that will be fulfilled by the clean (wind, solar photovoltaic) and dirty (coal, natural gas) technologies. I consider this an acceptable simplification that can still show the dynamics of the clean energy transition, though future work may want to include hydro and nuclear as well as separate wind, solar, gas and coal in to separate sectors. Energy demand growth is roughly linear over time, going from 2836 billion kWh in 2011 to 3734 billion kWh in 2040. For any simulation length greater than 30 years, I assume that growth will continue to grow at the same per year rate. 

\paragraph{} While there may be future improvements in the cost of coal and natural gas plants, we might expect the majority of capital cost per kilowatt-hour decreases to come from the continuing switch from coal to gas-fired generation over the next 50 years. According to the EIA, coal powered generating capacity will go from 1717 billion kWh/year in 2011 (or 43\% of net generating capacity) to 1661 billion kWh/year (34\% of net generating capacity) in 2040, while natural gas will go from 925 billion kWh/year (23\%) in 2011 to 1605 billion kWh/year (33\%) in 2040 \cite{EIATable5}. Also from the EIA are estimates of current overnight capital costs to build a kW of generating capacity \cite{EIACapCost}. I then compute a weighted average overnight capital cost from the projected generating capacity and the current overnight capital costs. Note that $F^h(t)$ is the inverse of this weighted average. To arrive at a generating capacity per year, I convert the kW generating capacity to kWh/year assuming a capacity factor of 50\%. Similarly, I assume a weighted average of carbon intensities of energy for coal and gas. As the US economy transitions to having more gas-fired power plants, the carbon intensity of energy will go down over time. A summary of these parameters can be found in Table \ref{tab:ParamValues}. 

\begin{figure}[h]\label{fig:FlScaleScenarios}
\includegraphics[width=0.9\textwidth]{Fl_Simulation_Range.png}
\caption{Scenarios of clean capital cost transition speed}
\end{figure}

\paragraph{} Of course, the main source of uncertainty in this model is the speed of cost reductions for clean energy capital. I model this forecast deterministically as a logistic function, with some initial low value and a maximum number of kWh/year per dollar that represents the cost floor of a mature technology. Varying the speed of transition between the minimum and maximum kWh/year/dollar allows us to understand the sensitivity of the objective function with respect to the speed by which clean capital cost decreases. Figure \ref{fig:FlScaleScenarios} gives some cost curve scenarios in comparison with the linear high-emitting capital cost curve. The minimum cost in each of these figures is half of the current weighted average overnight capital cost from the EIA and the maximum is the cost per kWh/year of natural gas. Both the minimum and the maximum are then adjusted to account for a lower renewable capacity factor (30\%) that accounts for the intermittent nature of renewable power generation. Emissions per kWh of clean energy are assumed to be 0; that is, any $C0_2$ emissions from the construction of the wind turbines or photovoltaic cells are assumed to be negligible. 


\begin{table}[h]\label{tab:ParamValues}
\begin{tabulary}{1.0\textwidth}{LLLL}
Variable & Description & Value\\
\hline
$\alpha$ &  \% Business As Usual Emissions  &  Varied \\
$Fl_min$ & Minimum kWh/year per dollar - Clean Capital & \\
$F_max$ & Maximum kWh/year per dollar - Clean Capital & \\
$Scale$ & Transition Speed  (Scale of Logistic Function) & \\
$Fh_min$ & Minimum kWh/year per dollar - Dirty Capital & \\
$Fh_max$ & Maximum kWh/year per dollar - Dirty Capital & \\
$G_0$ & Base energy demand (bn kWh/year) & \\
$G_m$ & Annual increase in energy demand (bn kWh/year) & \\
$c^l$ & Capacity Factor Clean Capital & 30\% \\
$c^h$ & Capacity Factor Dirty Capital & 50\% \\
\hline
\end{tabulary}
\caption{Key parameters of simulation. Note that energy demand is excluding the generating capacity of hydroelectric dams and nuclear plants}
\end{table}

As we can see from the parameters in this table, clean capital has a higher cost per kWh/year than dirty capital, yet it has a significantly lower carbon footprint. By forcing capital in each period to meet some exogenous energy demand, we force the model to reconcile these two competing parameters, leading to changes in optimal investment profiles that fulfill the modeling requirements.



\section{Model Results}
\FloatBarrier

\paragraph{} I run the model multiple sets of parameter values. Specifically, two parameters of interest are the speed of clean capital cost reductions (Scale in table \ref{tab:ParamValues}) and the fraction of business as usual emissions that define the emissions cap. Based on the equations in section \ref{sec:ModelDesc}, we would expect the minimum cost of the transition to increase with the size of the emissions cap. As we require a smaller and smaller percentage of $CO_2$ to be emitted over the length of the simulation, we must build more and more low-emitting (clean) capital. Since it is more costly to build per kilowatt-hour than coal or gas plants, our objective function is penalized. Figure \ref{fig:costVsAlpha} shows the relationship between emissions reduction and total transition cost. 

\begin{figure}[h]
\includegraphics[width=\textwidth]{cost_as_func_of_Alpha.png}
\caption{Cost vs Alpha\label{fig:costVsAlpha}}
\end{figure}



\paragraph{} For high values of $\alpha$, the emissions cap is small compared to business as usual - that is, we commit to 1-$\alpha$ reduction in BAU emissions. Consequently, we do not have to build large amounts of clean capital to accomplish such a reduction and because it is less cost-efficient to build, we chose not to invest heavily in clean technologies. Figure \ref{fig:SimpleResultsA=75} gives an example of such a transition where the emissions cap is set to emit no more than 75\% of business as usual emissions. 

\begin{figure}[h]
\includegraphics[width=\textwidth]{cap_and_invest_results_A_75.png}
\caption{Alpha = 75\label{fig:SimpleResultsA=75}}
\end{figure}



%%%%% Issue with references to figures. check if still issue later! %%%%%%%%%%%%%


\paragraph{} In comparison, when the emissions cap is set much lower, say 25\% of business as usual emissions, investment in clean technologies must happen much sooner and with a higher magnitude of investment. Such a transition is shown in figure \ref{fig:SimpleResultsA=25}. 

\begin{figure}[h]
\includegraphics[width=\textwidth]{cap_and_invest_results_A_25.png}
\caption{Alpha = 25\label{fig:SimpleResultsA=25}}
\end{figure}



\paragraph{} The characteristic spike in clean capital investment appears in all but the most extreme cases of parameter selection. Shown below are results from running the simulation over various speeds of clean capital cost reduction. As we would expect, the faster the cost reduction speed, the lower the cost of transition because we can take advantage of more cost-effective clean capital earlier in the simulation, reducing the need to early retire polluting capital once the economy nears it's emissions cap. The results from these simulations are shown in figure \ref{fig:costVsFl}. 

\begin{figure}[h]
\includegraphics[width=\textwidth]{cost_as_func_of_Fl_emissions_05.png}
\caption{Cost vs Fl\label{fig:costVsFl}}
\end{figure}


\begin{figure}[h]
\includegraphics[width=\textwidth]{cap_and_invest_results_Fl_1.png}
\caption{Fl = 1\label{fig:SimpleResultsFl=1}}
\end{figure}

\paragraph{} Similar patterns in our optimization solution emerge when we consider the speed of capital cost reductions of clean technologies. A low speed transition, such as one shown in figure \ref{fig:SimpleResultsFl=1} shows that additional positive investments in high-emitting technologies must be made early in the simulation to ensure adequate generating capacity. However, much of this early investment must be retired early to avoid exceeding the emissions cap resulting in a higher penalty to the overall transition cost as less cost-effective clean capital is substituted. This is contrasted with a very quick cost reduction scenario such as in figure \ref{fig:SimpleResultsFl=100} where hardly any additional high-emitting capital is invested at all. Instead the model can meet it's generation demand by substituting clean energy capacity at a relatively lower cost per kilowatt hour than in the previous scenario. While some high emitting capital must still be retired early, the amount of capital subjected to early retirement is roughly an order of magnitude less than in the slow-transition speed scenario. 

\begin{figure}[h]
\includegraphics[width=\textwidth]{cap_and_invest_results_Fl_100.png}
\caption{Fl = 100\label{fig:SimpleResultsFl=100}}
\end{figure}



\FloatBarrier

\section{Vintaged Model}

TBD


\subsection{Model Math}

Capital now is allowed to have a vintage associated with it's first year of construction:

\begin{equation}
k_i^h(t) = H_i^+ e^{-i-t/n} - \int_i^t H^-_i(x)dx
\end{equation}

\begin{equation}
k_i^l(t) = L_i^+ e^{-i-t/n} - \int_i^t L^-_i(x)dx
\end{equation}

\begin{equation}
H^+_i, H^-_i(t), L^+_i(t), L^-_i(t) \geq 0 \forall i, t
\end{equation}

min Cost (TBD) subject to:

\begin{equation}
G(t) = \int_0^t F^h(i)k_i^h(t) + F^l(i)k_i^l(t) di
\end{equation}

\begin{equation}
\int_{t=0}^{t=N}\int_{i=0}^{i=t} m^h(i)F^h(i)k_i^h(t) + m^l(i)F^l(i)k_i^l(t) di dt \leq \bar{E}
\end{equation}

\begin{equation}
H_i^+ e^{-i-t/n} - \int_i^t H^-_i(x)dx \geq 0
\end{equation}

\begin{equation}
L_i^+ e^{-i-t/n} - \int_i^t L^-_i(x)dx \geq 0
\end{equation}



Imagine that there is no negative investment for a particular vintage of high-emitting capital. Total lifetime capital would be:

\begin{equation}
C_i = \int_i^N H^+_i e^{i-t/n}dt
\end{equation}

In reality, there will be one or more disinvestments over time, such that the actual useful capital is defined by our capital equations as:

\begin{equation}
K^{total}_i = \int_i^N \left[H^+_i e^{i-t/n} dt - \int_i^t H^-_i(x)dx \right] dt
\end{equation}

The operating cost that is saved is a fraction $\beta$ of the difference of these two terms:

\begin{equation}
OC_i = \beta (C_i - K^{total}_i)
\end{equation}


Substituting this expression and integrating over all vintages  i $\in \{0,N\}$:

\begin{equation}
OC_i = \beta \int_{i=0}^N \int_{t=i}^N \int_{x=i}^t H^-_i(x) dx dt di
\end{equation}

So the total objective function is then:

\begin{equation}
C = \int_{i=0}^N  \left[ (H^+_i + L^+_i)e^{-ri}  - \beta  \int_{t=i}^N \int_{x=i}^t H^-_i(x)e^{-rx} dx dt\right] di
\end{equation}





\section{Model Discussion}

efficiencies/emissions and their problem
biases of results
operating costs 



%%%%%%%%%%%%%%%%%%%%%%%%%%%%%%%%%%%%%%%
%%%%%%%%%%%%%%%%%%%%%%%%%%%%%%%%%%%%%%%

\chapter{Conclusions}

Future work - learning by doing, uncertainty




\singlespacing
\pagebreak
\addcontentsline{toc}{chapter}{References}

\begin{thebibliography}{99}

\bibitem{EIATable5}
fill in



\bibitem{EIACapCost}
http://www.eia.gov/forecasts/capitalcost/pdf/updated\_capcost.pdf





\end{thebibliography}

\end{document}






